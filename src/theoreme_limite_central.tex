\chapter{Théorème limite centrale}

\label{chapter:theoreme_limite_centrale}

\section{Convergence en loi}

\begin{definition}
	Soit $(X_{n})_{n \in \naturel}$ une suite de variables aléatoires réelles et
	soit $X$ une autre variable aléatoire réelle.

	On dit que \textbf{$X_{n}$ converge en loi vers $X$} si
	pour toute fonction $\GSfunction{f}{\real}{\real}$ continue et bornée, on a

	\begin{equation}
		\int_{\real} f(x) dP_{X_{n}} \rightarrow \int_{\real} f(x) dP_{X}
	\end{equation}

	On dit aussi que \textbf{$X_{n}$ converge en distribution} ou \textbf{converge
	faiblement} ou encore \textbf{converge étroitement vers $X$}.
\end{definition}

\begin{remarque}
	Si $f(x) = e^{itx}$, on obtient
	\begin{align}
		\int_{\real} e^{itx} dP_{X_{n}} &\rightarrow \int_{\real} e^{itx} dP_{X}
	\end{align}
	C'est-à-dire
	\begin{align}
		\phi_{X_{n}}(t) &\rightarrow \phi_{X}(t)
	\end{align}

	Donc, si $X_{n}$ converge en loi vers $X$, alors la suite $\phi_{X_{n}}$
	converge simplement vers $\phi_{X}$.
\end{remarque}

\begin{remarque}
	\begin{enumerate}
		\item Les variables aléatoires $X_{n}$ ne sont pas forcément définies sur un même
		$\Omega$. Nous ne regardons que la loi à travers l'intégrale. Cela
		différencie cette notion de convergence par rapport aux autres car les
		autres demandaient que les variables de la suite soient définies sur le même
		$\Omega$.
		\item L'ensemble des fonctions continues bornées de $\real$ dans $\real$, noté
		$\mathcal{C}_{b}(\real)$ forment un espace de Banach.
	\end{enumerate}
\end{remarque}

\begin{remarque}
	On a $X_{n}$ converge en loi vers $X$ ssi pour toute fonction $f \in
	\mathcal{C}_{b} (\real)$

	\begin{equation}
		\int_{\Omega} f(X_{n}) dP_{X_{n}} \rightarrow \int_{\Omega} f(X) dP_{X}
	\end{equation}

	ou de manière équivalente, en se rappelant la définition d'espérance d'une
	variable aléatoire réelle

	\begin{equation}
		E(f(X_{n})) \rightarrow E(f(X))
	\end{equation}
\end{remarque}

Quel est le lien entre la convergence en loi et les autres ?

\begin{proposition}
	Soit $(X_{n})_{n \in \naturel}$ une suite de variables aléatoires réelles et
	$X$ une autre variable aléatoire réelle.

	Si $X_{n}$ converge en loi vers $X$, alors $X_{n}$ converge en probabilité
	vers $X$.
\end{proposition}

\ifdefined\outputproof
\begin{proof}

\end{proof}
\fi

\begin{remarque}
	La réciproque est fausse. En effet, si on prend la suite $X_{n} = (-1)^{n}$
	où $X$ est une variable aléatoire réelle de loi normale centrée réduite,
	alors $X_{n}$ converge en loi vers $X$ ainsi que $-X$. Or, la limite de la
	converge en proba est unique.

	En particulier, nous venons de remarquer à travers cet exemple que
	\textbf{la limite d'une suite qui converge en loi n'est pas unique}.
\end{remarque}

On obtient quand même quelque résultat pour des convergences spécifiques.

\begin{proposition}
	Soit $(X_{n})_{n \in \naturel}$ une suite de variables aléatoires réelles et
	$c$ une constante.

	Si $X_{n}$ converge en loi vers $c$, alors $X_{n}$ converge en probabilité
	vers $c$.
\end{proposition}

\ifdefined\outputproof
\begin{proof}

\end{proof}
\fi

Donnons une équivalence lorsque nous travaillons avec des variables aléatoires
réelles discrètes.

\begin{proposition}
	Soit $(X_{n})_{n \in \naturel}$ une suite de variables aléatoires réelles
	discrètes et $X$ une autre variable aléatoire réelle discrète.

	Alors, les assertions suivantes sont équivalentes.

	\begin{enumerate}
		\item $X_{n}$ converge en loi vers $X$.
		\item pour tout $k \in \naturel$, $P(X_{n} = k) \rightarrow P(X = k)$
	\end{enumerate}
\end{proposition}

\ifdefined\outputproof
\begin{proof}

\end{proof}
\fi

Nous avons remarqué précédemment que si une suite $X_{n}$ convergeait
en loi vers $X$, alors la suite des fonctions caractéristiques des $X_{n}$
convergeait simplement vers la fonction caractéristique de $X$.

En réalité, la condition est suffisante, et nous obtenons donc le théorème suivant.

\begin{theorem} [Paul-Lévy]
	Soit $(X_{n})_{n \in \naturel}$ une suite de variables aléatoires réelles et
	$X$ une autre variable aléatoire réelle.

	Alors, les assertions suivantes sont équivalentes.

	\begin{enumerate}
		\item $X_{n}$ converge en loi vers $X$.
		\item la suite des fonctions caractéristiques $\phi_{X_{n}}$ convergent
			simplement vers $\phi_{X}$, la fonction caractéristique de $X$.
	\end{enumerate}
\end{theorem}

\ifdefined\outputproof
\begin{proof}

\end{proof}
\fi

Donnons une autre équivalence, mais par rapport aux fonctions de répartition.

\begin{theorem}
	Soit $(X_{n})_{n \in \naturel}$ une suite de variables aléatoires réelles et
	$X$ une autre variable aléatoire réelle \textbf{tel que $F_{X}$ est
continue}.

	Alors, les assertions suivantes sont équivalentes.

	\begin{enumerate}
		\item $X_{n}$ converge en loi vers $X$.
		\item la suite des fonctions de répartition $F_{X_{n}}$ convergent simplement
			vers $F_{X}$, la fonction de répartition de $X$.
	\end{enumerate}
\end{theorem}

\ifdefined\outputproof
\begin{proof}

\end{proof}
\fi

Cependant, nous pouvons restreindre l'hypothèse de continuité en certains
points.

\begin{proposition}
	Soit $(X_{n})_{n \in \naturel}$ une suite de variables aléatoires réelles et
	$X$ une autre variable aléatoire réelle tel que $X_{n}$ converge en loi vers
	$X$.

	\begin{enumerate}
		\item Si $F_{X}$ est continue en $x$, alors $P(X_{n} \leq x) \rightarrow
			P(X \leq x)$.
		\item Si $F_{X}$ est continue en $x$, alors $P(X_{n} < x) \rightarrow
			P(X < x)$.
		\item Si $F_{X}$ est continue en $a$ et $b$, alors $P(X_{n} \in [a, b])
			\rightarrow P(X \in [a, b])$.
	\end{enumerate}

	En particulier, on a les assertions suivantes qui sont équivalentes,
	découlant de cette
	proposition et de la précédente.

	\begin{enumerate}
		\item $X_{n}$ converge en loi vers $X$.
		\item pour tout $a, b \in \real$, $P(X_{n} \in [a, b]) \rightarrow P(X \in [a,
			b])$.
	\end{enumerate}
\end{proposition}

\ifdefined\outputproof
\begin{proof}

\end{proof}
\fi

\section{Enoncé du théorème de limite central}

\begin{theorem}
	Soit $(X_{n})_{n \in \naturel}$ une suite de variables aléatoires réelles
	iid.

	Posons $E(X.) = \mu$, $Var(X.) = \sigma^{2}$ et
	\begin{equation}
		S_{n} = \sum_{i = 1}^{n} X_{n}
	\end{equation}

	Alors, la suite

	\begin{equation}
		\overset{\sim}{S_{n}} := \frac{S_{n} - n \mu}{\sigma \sqrt{n}}
	\end{equation}

	converge normalement vers $\lawNormal{0}{1}$, la loi normale centrée réduite.
\end{theorem}

\ifdefined\outputproof
\begin{proof}

\end{proof}
\fi
