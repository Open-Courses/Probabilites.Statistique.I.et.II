\chapter{Variables aléatoires réelles}

\label{chapter:variables_aleatoires_reelles}

\section{Calculs de lois}

\begin{proposition}
	Soit $X : \Omega \rightarrow \real$ une variable aléatoire réelle de loi
	$P_{X}$.

	Soit $\mu : \borelian{\real} \rightarrow [0, 1]$ une probabilité
	tel que, pour toute fonction $h : \real \rightarrow \real$ mesurable
	positive,

	\begin{equation}
		E(h(X)) = \int_{\real} h(x) d\mu
	\end{equation}

	Alors $\mu = P_{X}$.
\end{proposition}

\ifdefined\outputproof
\begin{proof}

\end{proof}

\begin{exercice}
	Soit $X$ une variable aléatoire réelle. Calculer la loi de $Y$ pour les cas
	suivants.
	\begin{itemize}
		\item $X \varFollow \lawNormal{0}{1}$ et $Y := X^{2}$.
		\item $X \varFollow \lawUniform{[0, 1]}$ et $Y := \max(X, -X)$.
		\item $X \varFollow \lawCauchy$ et $Y := \frac{1}{X}$.
	\end{itemize}
\end{exercice}
